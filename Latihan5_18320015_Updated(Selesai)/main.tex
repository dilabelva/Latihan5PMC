%-----Latihan 5-----
%Nama    : Atadila Belva Ganya
%NIM     : 18320015
%Jurusan : Teknik Biomedis'20
%Program Last Update on Monday, 21 February 2022, 18.02 WIB

\documentclass[conference]{IEEEtran}
\usepackage{cite}
\usepackage{hyperref}
\hypersetup{
    colorlinks=true,
    linkcolor=black,     
    urlcolor=blue,
    citecolor=black, 
} 
\usepackage{smartdiagram}  
\usepackage{graphicx}
\usepackage[ruled,vlined]{algorithm2e}
\usepackage{algpseudocode}
\usepackage{amsmath,amssymb,amsfonts}
\let\oldemptyset\emptyset
\let\emptyset\varnothing

%Judul
\title{Implementasi Algoritma Dijkstra Dalam\\
Menemukan Jarak Terdekat Dari Lokasi Pengguna\\
Ke Tanaman Yang Di Tuju}

%Penulis
\author{\IEEEauthorblockN{Atadila Belva Ganya}
\IEEEauthorblockA{\textit{School of Electrical Engineering and Informatics}\\
\textit{Institut Teknologi Bandung}\\
Bandung, Indonesia\\
Email: 18320015@std.stei.itb.ac.id}
}

% folder gambar
\graphicspath{{./gambar/}}

\begin{document}

\maketitle

\begin{abstract}
    Kebun Raya Purwodadi dengan luas area sekitar 85
    hektar ternyata kekurangan papan informasi yang menyebabkan
    pengunjung kerap kali kebingungan dalam mencari lokasi tana-
    man tertentu. Paper ini bertujuan untuk membuat simulasi
    dari algoritma yang dapat menentukan jarak terdekat antara
    pengunjung (pengguna program) dengan lokasi tanaman yang
    dituju. Algoritma yang digunakan adalah algoritma Dijkstra
    yang beroperasi secara menyeluruh (\emph{greedy}) untuk menguji
    seitap persimpangan (\emph{Vertex}) dan jalan (\emph{Edge}) pada Kebun
    Raya Purwodadi. Berdasarkan hasil simulasi dan pengujian,
    kompleksitas ruang dari program ini adalah O(V) karena adanya
    pembentukan array yang berisi V \emph{nodes} untuk mencari \emph{heap} min-
    imum. Sementara, kompleksitas waktu dari algoritma tersebut
    adalah O(V$\sp{2}$).
\end{abstract}

\begin{IEEEkeywords}
    Dijkstra, \emph{Vertex}, \emph{Edge}, Tanaman.
\end{IEEEkeywords}

\section{Introduction}
\setlength{\intextsep}{5pt} 
    Studi mengenai penggunaan algoritma Dijkstra dalam men-
    cari  jarak  terdekat  dapat  diimplementasikan  pada  kasus  pen-
    carian tanaman pada Kebun Raya Purwodadi seperti yang telah
    dilakukan oleh Yusuf et al di tahun 2017~\cite{j-ptiik}. Paper ini bertu-
    juan  untuk  melakukan  simulasi  kembali  terhadap  peneliti
    anyang  telah  dilakukan  dengan  bahasa  C  serta  mengevaluasi
    efisiensinya  melalui  perhitungan  kompleksitas  waktu  dan  ru-
    ang dengan analisis Big-O.\par

    Di  Kecamatan  Purwodadi,  Kabupaten  Pasuruan,  terdapat
    salah  satu  kebun  raya  di  Indonesia  yang  bernama  Kebun
    Raya  Purwodadi  yang  memiliki  luas  area  hingga  85  hektar.
    Kebun raya sebagai fasilitas rekreasi dan penelitian ini ternyata
    kekurangan papan informasi yang seharusnya disediakan oleh
    pihak pengelola. Hal ini menyebabkan banyaknya pengunjung
    yang merasa kebingungan untuk mencari lokasi dari tanaman
    tertentu.  Oleh  karena  itu,  Yusuf  et  al  (2017)  memutuskan
    untuk membuat suatu aplikasi dengan memanfaatkan algoritma
    Dijkstra untuk membantu pengunjung Kebun Raya Purwodadi
    dalam mencari lokasi tertentu.\par

    Algoritma Dijkstra digunakan karena algoritma ini dapat
    beroperasi secara menyeluruh (algoritma \emph{greedy}) terhadap
    semua alternatif fungsi serta durasi eksekusi yang lebih cepat
    jika  dibandingkan  dengan  algoritma  serupa,  yaitu  Bellman
    -Ford.  Algoritma  ini  akan  mencari  jalur  dengan  ’biaya’  atau
    cost terendah antara dua titik dengan membandingkan semua
    alternatif yang ada.\par

    Pada kasus ini, masing-masing persimpangan di Kebun
    Raya Purwodadi direpresentasikan sebagai \emph{vertex} dan setiap
    jalan direpresentasikan sebagai \emph{edge}. Rute terdekat yang dida-
    patkan akan diperoleh dari pembobotan setiap \emph{vertex} dan \emph{edge}
    berdasarkan  jarak  antara  titik  pengguna  dengan  titik  tujuan
    atau tanaman.\par

\section{Studi Pustaka}

\subsection{Algoritma Dijkstra}

\RestyleAlgo{ruled}
\begin{algorithm}
    \SetAlgoLined
    \DontPrintSemicolon
    \caption{Dijkstra's Algorithm \texttt{Dijkstra}}
    \KwResult{Find the shortest path from a to z}
    \textbf{procedure} \emph{Dijkstra}(\emph{G}: weighted connected simple
    graph, with all weights positive)\\
    \{\emph{G} has vertices $a = v_0, v_1, ..., v_n = z$ and lengths
    $w$($v_i, v_j$) where $w$($v_i, v_j$) $= \infty$ if $v_i, v_j$ is not an
    edge in \emph{G}\}\\
    \For{$i := 1$ \KwTo $n$}{
        $L$($v_i$)$ := \infty$
    }
    $L$($a$)$ := 0$\\
    $S := \oldemptyset$\\
    \{the labels are now initialized so that the label of $a$ is
    0 and all other labels are $\infty$, and $S$ is the empty set\}\\
    \While{$z \notin S$}{
        $u:=$ a vertex not in S with $L$($u$) minimal\\
        $S:=S \cup \{u\}$\\
        \For{\emph{all vertices v not in S}}{
            \If{$L$($u$)$+w$($u,v$)$<L$($v$)}{
                $L$($v$)$:=L$($u$)$w$($u,v$)\\
                \{this adds a vertex to $S$ with minimal label
                and updates the labels of vertices not in S\}
            }
        }
    }
   \Return $L$($z$) = \textit{length of a shortest path from a to z}%
\end{algorithm}
\setlength{\intextsep}{5pt} 
    Algoritma Dijkstra adalah algoritma yang digunakan untuk
    menemukan  jarak  jalur  terpendek  antara  dua \emph{vertice} pada
    \emph{graph} berbobot  dan  tidak  berarah  sederhana~\cite{rosen-discrete-2012}. Berbobot
    berarti grafik memiliki \emph{edge} dengan suatu ’bobot’ atau harga.
    Bobot  dapat  merepresentasikan  jarak,  waktu,  atau  apapun
    yang  memodelkan  koneksi  antara  kedua \emph{node}.  Tidak  berarah
    memiliki  arti  bahwa  untuk  setiap \emph{node} yang  terhubung,  kita
    dapat mendekati suatu \emph{node} dari kedua arah. Pendekatan Di-
    jikstra juga memiliki asumsi bahwa bobot pada \emph{edge} memiliki
    nilai  yang  tidak  negatif.  Hal  ini  karena  nilai  bobot  akan
    terus  dibandingkan  dan  diambil  nilai  yang  paling  kecil.  Ada
    banyak  varian  pada  algoritma  ini,  namun  pada  percobaan
    ini  digunakan  varian  dimana  suatu \emph{node} ditetapkan  menjadi
    \emph{source node}. Dari \emph{node} inilah  akan  dicari  jarak  terpendek
    diantara \emph{node} lain.  Algoritma  ini  dicetuskan  oleh  Edsger
    Wybe  Dijkstra,  salah  seorang  tokoh  ternama  di  bidang \emph{com-
    puter} science~\cite{dijkstra-note-1959}. Kompleksitas dari algoritma dijkstra adalah
    \emph{O}($n\sp{2}$), dengan \emph{n} menyatakan jumlah \emph{vertice} dari \emph{graph} yang
    bersangkutan.

\subsection{Kebun Raya Purwodadi}
    Kebun  Raya  Purwodadi  adalah  kebun  penelitian  di  Keca-
    matan  Purwodadi,  Jawa  Timur.  Ia  juga  dikenal  dengan  nama
    Hortus Ilkim Kering Purwodadi dan didirikan tanggal 30 Jan-
    uari 1941 oleh Dr. L.G.M. Baas Becking. Sebagai cabang dari
    Kebun Raya Bogor, ia memiliki fungsi mengkoleksi tumbuhan
    yang hidup di dataran rendah kering. Sebagai Balai Konservasi
    Tumbuhan di bawah Pusat Konservasi Tumbuhan Kebun Raya,
    Kedeputian Bidang Ilmu Pengetahuan Hayati LIPI, kebun raya
    ini  memiliki  banyak  tumbuhan  yang  dinaunginya.  Dengan
    menggunakan  algoritma  Dijkstra,  diharapkan  ia  dapat  mem-
    bantu  pengunjung  mencari  tanaman  tertentu  maupun  jara
    kyang paling optimal.

\section{Metodologi Penelitian}
    Peneliti  menggunakan  beberapa  tahap  dalam  penyusunan
    paper  ini.  Pertama,  dilakukan  pengkajian  dan  studi  literatur
    dengan membaca referensi paper yang berkaitan dan memilih
    paper  yang  dapat  menjadi  acuan  dalam  penelitian  yang  di-
    lakukan, sehingga dari pilihan topik dan tema yang berkaitan
    secara luas dapat dikecilkan menjadi sebuah paper yang men-
    cakup  mayoritas  dari  topik  yang  dibahas.  Setelah  ditemukan
    beberapa  paper,  dilakukan  perangkuman  untuk  menentukan
    paper  yang  sesuai  sekaligus  membahas  poin-poin  penting
    dari  paper  yang  ingin  dicapai.  Setelah  kedua  tahap  tersebut
    dilewati, penentuan paper yang dijadikan prototype penelitian
    merupakan  hal  yang  mudah  dan  menjadi  titik  pencapaian
    dalam studi literatur dan pemilihan topik dari prototype peneli-
    tian yang dilakukan.\par

    Setelah  itu,  tahap  selanjutnya  yang  dilakukan  oleh  peneliti
    adalah   pembuatan   prototype   berupa   program   yang   ditulis
    dalam  bahasa  C.  Pembuatan  prototype  berupa  kode  ini  di-
    lakukan terus-menerus dengan menggunakan metode trial and
    error  sehingga  perlu  dilakukan  revisi  hingga  protoype  kode
    yang  dibuat  dapat  mendapatkan  output  yang  optimal  dan
    sesuai  dengan  spesifikasi  yang  diharapakan.  Tahap  terakhir
    dari   penelitian   adalah   pemaparan   kode   yang   berhasil   di-
    jalankan tersebut ke dalam paper.

\tikzset{priority arrow/.append style={rotate=180,anchor=0,xshift=30,}} %rotating arrow for 180 degree    
\smartdiagram[priority descriptive diagram]{  
    Penyusunan Laporan,  
    Finalisasi Prototype,  
    Melakukan Revisi Prototype,  
    Membuat Prototype dalam Bahasa C,  
    Membuat Desain/Algoritma Prototype,  
    Memilih Topik Paper yang Akan Dibuat Prototypenya,
    Merangkum Paper,
    Membaca beberapa paper}

\section{Implementasi dan Pengujian}

\subsection{Implementasi Graph pada Array dalam Bahasa C}
    Program  akan  dimulai  dengan  pembacaan  file  bernama
    \emph{listtanaman.txt}. File tersebut akan menyimpan informasi men-
    genai semua nama tanaman yang bersangkutan. Setelah pem-
    bacaan tersebut, akan dicari informasi mengenai bobot graph
    yang menghubungkan \emph{node}. Informasi ini disimpan di dalam
    matriks  segitiga  bawah  kiri  didalam  file \emph{jarakantarpohon.txt}
    yang juga dibuka saat program dijalankan. Matriks menggam-
    barkan bobot antara jarak dua \emph{node} tanaman sekali saja karena
    pemodelan \emph{undirected graph} yang  memiliki jarak  sama  baik
    dari \emph{a} ke \emph{b} maupun \emph{b} ke \emph{a}.  Nilai --1 akan menggambarkan
    bagian \emph{node} yang  tidak  terhubung  sama  sekali  dalam  graph
    dan  juga  dinyatakan  dalam  suatu  variabel  bernama  int\_max
    (Jaraknya  sebesar  tak  hingga).  Nilai  jarak  terpendek  akan
    disimpan dalam array tersebut selagi program berjalan.

\subsection{Implementasi Algoritma Dijkstra dalam Bahasa C}
    
    Dalam  implementasi  algoritma,  abstraksi  dengan  menggu-
    nakan pseudocode dapat dibagi menjadi dua buah fungsi dan
    satu  program  utama.  Fungsi  yang  digunakan  adalah  fungsi
    printgraph (Fungsi Graph) untuk memunculkan graph beruku-
    ran \emph{n} x \emph{n} ke  layar  pengguna.  Algoritma  dari  fungsi  tersebut
    dapat dilihat pada bagian di bawah ini:\par

    \RestyleAlgo{ruled}
    \begin{algorithm}
        \SetAlgoLined
        \DontPrintSemicolon
        \caption{Fungsi Graph \texttt{(printgraph)}}
        \KwResult{Memunculkan Graph $n \times n$ Ke Layar}
        \textbf{procedure} \emph{printgraph(n, graph[n][n])}\\
        \While{$j \le n - 1$}{
            $j \gets 0;$\;
            \While{$j \le n - 1$}{
                \eIf{$graph[i][j]=int\_max$}{
                    \textbf{output} $(-1)$;\;
                }{\textbf{output} $graph[i][j]$;\;}
                $j \gets j + 1;$\;
            }
            $i \gets i + 1;$\;
        }
    \end{algorithm}

    Fungsi kedua yang digunakan adalah fungsi pencari indeks
    pada  array  yang  akan  diproses  dengan  menggunakan  pen-
    dekatan  algoritma  Dijkstra.  Abstraksi  fungsi  yang  digunakan
    dapat dilihat pada bagian berikut ini:

    \RestyleAlgo{ruled}
    \begin{algorithm}
        \SetAlgoLined
        \DontPrintSemicolon
        \caption{Fungsi Pencari Indeks \texttt{idx\_process}}
        \KwResult{Mencari indeks yang akan diproses dengan algoritma Dijkstra}
        \textbf{Initialization:}\\
        $is\_found \gets FALSE;$\;
        $i \gets 0;$\;
        \textbf{Algorithm:}\\
        \While{$j \le n - 1$}{
            $j \gets 0;$\;
            \If{$!is\_final[i]$ \textbf{and} $!is\_found$}{
                $idx\_min \gets i;$\;
                $val\_minimum \gets jarak\_f[i];$\;
                $is\_found \gets true;$\;
            }
            \If{$is\_found$ \textbf{and} $!is\_final[i]$ \textbf{and} $(jarak\_f[i]<val\_minimum)$}{
                $idx\_min \gets i;$\;
                $val\_minimum \gets jaral\_f[i];$\;
            }
        }
        \eIf{$is\_found$}{
            \Return $(idx\_min)$\;
        }{
            \Return $(int\_max)$\;
        }
    \end{algorithm}
    \setlength{\intextsep}{0pt} 
    Program   utama   akan   membaca   file   database   tanaman
    beserta  jarak  masing-masing  tanaman  dan  akan  mencetak
    daftar   tanaman   yang   berada   di   Kebun   Raya   Purwodadi.
    Kemudian, program akan menerima input salah satu tanaman
    terdekat dari pengguna sebagai penanda posisi awal pengguna.
    Setelah  itu,  program  akan  kembali  menerima  input  posisi
    tanaman tujuan dan memproses pencarian rute terdekat dengan
    algoritma  Dijkstra.  Rute  yang  diperlukan  akan  ditampilkan
    dalam bentuk list nama tanaman yang harus dilalui pengguna
    dan   menampilkan   jarak   antara   kedua   tanaman   tersebut.
    Implementasi algoritma dalam abstraksi tersebut dapat dilihat
    pada gambar di bawah ini:
    \RestyleAlgo{ruled}
    \begin{algorithm}
        \SetAlgoLined
        \DontPrintSemicolon
        \caption{Program Utama Pencarian Rute Antara Dua Tanaman - Pembacaan Jumlah Tanaman}
        \KwResult{Menyimpan nama tanaman pada sebuah array}
        \textbf{Algorithm:}\\
        \textbf{input}(\texttt{namafile\_tanaman});\;
        \textbf{open}(\texttt{namafile\_tanaman});\;
        \textbf{read}(\texttt{namafile\_tanaman});\;
        \texttt{n\_tanaman} $\gets 0;$\;
        \texttt{baris} $\gets 0;$\;
        \While{\emph{\texttt{baris}} $\le$ \emph{\texttt{max\_len}}}{
            \texttt{token} $\gets$ \texttt{parse(baris)};\;
            \texttt{token} $\gets$ \texttt{nama\_tanaman[n\_tanaman]};\;
            \texttt{n\_tanaman} $\gets$ \texttt{n\_tanaman+1};\;
            \texttt{baris} $\gets$ \texttt{baris$+1$};\;
        }
    \end{algorithm}

    \setlength{\intextsep}{5pt} 
    Setelah  pembacaan  jumlah  tanaman  dari  file,  maka  diper-
    lukan graph atau jarak antar tanaman yang akan menjadi dasar
    perhitungan dari pencarian rute terdekat. Proses memasukkan
    graph dapat dilihat pada algoritma berikut ini:
    \RestyleAlgo{ruled}
    \begin{algorithm}
        \SetAlgoLined
        \DontPrintSemicolon
        \caption{Program Utama Pencarian Rute Antara Dua Tanaman - Memasukkan Graph}
        \KwResult{Menyimpan graph dalam sebuah matriks $n \times n$}
        \textbf{input}(\texttt{namafile\_graph});\;
        \textbf{open}(\texttt{namafile\_graph});\;
        \textbf{read}(\texttt{namafile\_graph});\;
        $baris \gets 0;$\;
        \While{$baris \le max\_len$}{
            $k \gets 0;$\;
            $token \gets parse(baris);$\;
            \While{$token != NULL$}{
                \texttt{graph[j][k]} $\gets token$;\;
                \texttt{graph[k][j]} $\gets token$;\;
                \eIf{$token == -1$}{
                    \texttt{graph[j][k]} $\gets$ \texttt{int\_max};\;
                    \texttt{graph[k][j]} $\gets$ \texttt{int\_max};\;
                }{
                    $k \gets k + 1$;\;
                    $token \gets parse(NULL)$;\;
                }
            }
            $baris \gets baris + 1$;\;
        }
    \end{algorithm}

    Setelah  data  yang  dibutuhkan  dimasukkan,  implementasi
    dari  algoritma  Dijkstra  untuk  pencarian  rute  terdekat  adalah
    sebagai berikut:
    \setlength{\intextsep}{0pt} 
    \RestyleAlgo{ruled}
    \begin{algorithm}
        \SetAlgoLined
        \DontPrintSemicolon
        \caption{Program Utama Pencarian Rute Antara Dua Tanaman: Pencarian Jarak dengan Algoritma Di-jkstra}
        \textbf{Algorithm:}\\
        \textbf{input}(\texttt{idx\_a});\;
        \texttt{idx\_a $\gets$ idx\_a$-1$};\;
        \textbf{input}(\texttt{idx\_tujuan});\;
        \texttt{idx\_tujuan $\gets$ idx\_tujuan$-1$};\;
        \For{\textit{\texttt{i} = 0 \textbf{to} \texttt{n\_tanaman}}}{
            \eIf{\textit{\texttt{i = idx\_a}}}{
                \texttt{jarak\_f[i] $\gets 0$};\;
                \texttt{is\_final[i] $\gets FALSE$};\;
            }{
                \texttt{jarak\_f[i] $\gets$ int\_max};\;
                \texttt{is\_final[i] $\gets FALSE$};\;
            }
            \For{\textit{\texttt{j = 0} \textbf{to} \texttt{n\_tanaman}}}{
                \texttt{list\_dilalui[i][j] $\gets$ int\_max};\;
            }
            \texttt{idx\_lalui[i] $\gets$ 0};\;
        }

        \texttt{jarak\_f[idx\_a] $\gets 0;$\;
        list\_dilalui[idx\_a][0] $\gets$ idx\_a$;$\;
        idx\_lalui[idx\_a] $\gets$ idx\_lalui[idx\_a]$+1;$\;
        idx\_now $\gets$ idx\_a$;$}\;

        \While{\textit{\texttt{idx\_now$!=$int\_max}}}{
            \texttt{is\_final[idx\_now] $\gets$}\emph{TRUE};\;
            \For{\textit{\texttt{i = 0} \textbf{to} \texttt{n\_tanaman}$-1$}}{
                \If{\textit{\texttt{(!is\_final[i])} \textbf{and} \texttt{graph[idx\_now][i]$!=$int\_max} \textbf{and} \texttt{(jarak\_f[idx\_now] $+$ graph[idx\_now][i]) $>$ jarak\_f[i]}}}{
                    \texttt{jarak\_f[i] $\gets$ (jarak\_f[idx\_now] $+$ graph[idx\_now][i]);\\
                    idx\_lalui[i] $\gets$ idx\_lalui[idx\_now]$+$1};\;
                }
                \For{\textit{\texttt{j = 0} \textbf{to} \texttt{idx\_dilalui[i]$-$1}}}{
                    \eIf{\textit{\texttt{j=idx\_dilalui[i]$-$1}}}{
                        \texttt{list\_dilalui[i][j]$\gets$i};\;
                    }{
                    \texttt{list\_dilalui[i][j] $\gets$ list\_dilalui[idx\_now][j]};\;
                    }
                }
            }
            \texttt{idx\_now $\gets$ idx\_process(n\_tanaman, jarak\_f, is\_final)};\;
        }
    \end{algorithm}
    \setlength{\intextsep}{5pt} 
\subsection{Implementasi Program dalam Bahasa C}
    Implementasi program dalam bahasa C dapat dilihat
    pada \emph{repository} berikut. \url{https://github.com/ReynaldoAverill/Tugas7PMC}
    
\subsection{Perhitungan Kompleksitas Waktu}
    Kompleksitas dari program ini dengan notasi kompleksitas
    Big  O  adalah \emph{O}($n\sp{2}$).  Hal  tersebut  disebabkan  pada  loop
    program  bagian \emph{for},  terdapat  loop \emph{for} lain  yang  berjumlah
    dua loop (Terletak pada bagian \emph{assign} kondisi awal dan ketika
    program menjalankan algoritma Djikstra). Karena hal tersebut,
    akibatnya adalah kompleksitas waktu akan naik seiring dengan
    naiknya \emph{n} program  yang  dijalankan,  namun  tidak  bersifat
    linear sehingga kompleksitas waktunya adalah \emph{O}($n\sp{2}$). Grafik
    kompleksitas waktu dapat direpresentasikan pada gambar 1.
    \setlength{\intextsep}{0pt} 
\begin{figure}[htbp]
    \centering
    \scalebox{0.25}{\input{gambar/picture.pdf_tex}}
    \caption{Kompleksitas Waktu Program}
\end{figure}

\subsection{Perhitungan Kompleksitas Tempat}
    Matriks  penyimpanan  yang  digunakan  pada  program  ini
    memiliki  ukuran  terbesar \emph{n} x \emph{n},  dengan  nilai \emph{n} merepresen-
    tasikan  banyak  tanaman  dalam  \emph{filelisttanaman.txt}.  Program
    akan  melalui  grafik  dan  menyimpan  nilai  bobot  antara \emph{node}
    sebesar matriks di atas, mengakibatkan program dengan kom-
    pleksitas \emph{O}($n\sp{2}$). Hal ini dapat dilihat pada grafik kompleksitas
    tempat di gambar 2.
    \setlength{\intextsep}{0pt} 
\begin{figure}[htbp]
    \centering
    \scalebox{0.25}{\input{gambar/picture.pdf_tex}}
    \caption{Kompleksitas Tempat Program}
\end{figure}
\setlength{\intextsep}{5pt} 

\section{Kesimpulan}
    Pada perhitungan Jarak Terdekat dalam suatu lokasi atau ru-
    ang dapat diimplementasikan penggunaan Algoritma Djikstra
    dalam perhitungannya untuk mencapai suatu target pada ruang
    tersebut  dari suatu  titik.  Terbukti dari  penelitian Kebun  Raya
    Purwodadi untuk menentukan Tanaman yang ingin dituju.

%Referensi
\bibliographystyle{IEEEtran}
\bibliography{references.bib}

\end{document}